\section*{Laboratory work \#2}
\phantomsection

\section{Purpose of the laboratory}
Gain knowledge about child windows and basics of working with keyboard.
\section{Laboratory Work Requirements}
\begin{itemize}
\item \textbf{Basic Level (grade 5 - 6) you should be able to:}
	\begin{enumerate}
	\item Create an animation based on Windows timer which involves at least 5 different drawn objects
      \end{enumerate}
\item \textbf{Normal Level (grade 7 - 8) you should be able to:}
      \begin{enumerate}
     \item Realize the tasks from Basic Level.
    \item Increase and decrease animation speed using mouse wheel/from keyboard
    \item Solve flicking problem describe in your readme/report the way you had implemented this
          \end{enumerate}
\item \textbf{Advanced Level (grade 9 - 10) you should be able to:}
      \begin{enumerate}
    \item Realize the tasks from Norma Level without Basic Level
    \item Add 2 animated objects which will interact with each other. Balls that have different velocity and moving angles. They should behave based on following rules:
    	 -At the beginning you should have 3 balls of different colours of the same size
		 -On interaction with each other, if they are of the same class (circle, square), they should change their color and be multiplied.
		 -On interaction with the right and left wall (the margins of the window), they should be transformed into squares.
		 -On interaction with the top and bottom of the window - the figures should increase their velocity.
		 -Please, take into consideration that the user can increase and decrease animation speed using mouse wheel/from keyboard
          \end{enumerate}
\item \textbf{Bonus point task:}
      \begin{enumerate}
    \item Use a scroll bar to scroll through application working space. Scroll should appear only when necessary (eg. when window width is smaller than 300px)
    	\end{enumerate}
  \end{itemize}  

\clearpage